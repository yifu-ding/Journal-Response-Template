% Response to Associate Editor
\AssociateEditor
\begin{generalcomment}
The reviewer appreciate the relevance of the work. They hint to a few aspects which should be addressed in a revision:
\begin{itemize}
    \item Please argue why you ... 
    \item The presentation could be more concise ... 
    \item The novelty ...   
    \item The experimental evaluation ...
    \item How about ...
\end{itemize}
\end{generalcomment}
\begin{revmeta}[]
We would like to extend my sincere gratitude for your thoughtful suggestions. We understand the importance of addressing these concerns about ... to enhance the quality and impact of our research. Below, we respond to each issue in turn.  
\end{revmeta}

\begin{revcommentToAssociateEditor}
Please argue why you ... 
\label{sec:why-use-bert}
\end{revcommentToAssociateEditor}
\begin{revmeta}[]
    \lipsum[1] \cite{gawlikowski2023survey,bi2018empirical,johnson2019survey}
\end{revmeta}



\begin{revcommentToAssociateEditor}
The presentation could be more concise ...
\end{revcommentToAssociateEditor}
\begin{revmeta}[]
    \lipsum[2] \cite{zhang2021videolt,ganganwar2012overview,cao2019learning}
\end{revmeta}


\begin{revcommentToAssociateEditor}
The novelty ...
\end{revcommentToAssociateEditor}
\begin{revmeta}[]
    \lipsum[3]
\end{revmeta}

\begin{revcommentToAssociateEditor}
Additional References and Comparison Baselines. 
\end{revcommentToAssociateEditor}
\begin{revmeta}[]
    \lipsum[4]
\end{revmeta}


\begin{revcommentToAssociateEditor}
The experimental evaluation ...
\end{revcommentToAssociateEditor}
\begin{revmeta}[]
    \lipsum[5]
\end{revmeta}

\clearpage
\printbibliography[heading=bibliography, title={References}, section=\therefsection]
\markboth{}{}
